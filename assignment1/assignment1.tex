\documentclass[11pt]{article}
\usepackage[letter paper, portrait, margin=1in]{geometry}


\begin{document}
\begin{center}
	\textbf{\Large{ASSIGNMENT 1}}
\end{center}
THE FTIC.csv file contains data for 9218 first-time college freshman.  The goal of this assignment is to explore the effect of increased admissions standards on enrollment and retention.
\begin{enumerate}
	\item Import the FTIC.csv file as a pandas data frame
	\item Find the dimension of the data frame, and print the first few rows of the data frame to begin inspecting the data.
	\item The column ``$2^{nd}$\underline{~~}FALL'' shows whether each student was retained until their second fall semester.  Store that column of data in a vector called retention.  Check the length of retention, view the first few elements of the vector, and table the vector.
	\item Store the PERCENTILE and SAT columns in vectors called rank and sat, create histograms, summaries, and find the mean of these vectors.
	\item Replace the `Y`/`N` values of retention with 1's and 0's, and find the total number of students retained, and the retention rate.
	\item Basic statistics:
		\begin{enumerate}
			\item Plot rank vs sat
			\item Create a linear regression model for predicting sat using rank, and summarize that model
			\item Plot retention vs rank
		\end{enumerate}
	\item Determining how to predict retention accurately will allow a school to admit students with the highest chance of being retained, and therefore increase retention rate.
		\begin{enumerate}
			\item Create a logistic regression model for predicting retention using rank.  
			\item Find the $p$-value for rank in the model above.
			\item Create a logistic regression model for predicting retention using sat.
			\item Find the $p$-value for sat in the model above.
			\item Create a logistic regression model for predicting retention using rank and sat.
			\item Find the $p$-value for sat and rank in the model above.
		\end{enumerate}
	\item 
\end{enumerate}

\end{document}
